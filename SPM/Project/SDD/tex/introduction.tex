\chapterLabel{Introduction}

This section gives a scope description and overview of everything included in this SRS document. Also, the purpose for
this document is described and a list of abbreviations and definitions is provided. It follows the IEEE standard for
Software Requirements Specification documents.

This section ends with an overview about how the rest of the document is organised to help people reading this document
have a better understanding of the document.

\sectionLabel{Purpose}

The purpose of this document is to give a detailed descript ion of the requirements for the ``\projectTitle{}'' software. It will illustrate the purpose and complete declaration for the development of system. It will also
explain system constraints, interface and interactions with other external applications. This document is primarily
intended to be proposed to a customer for its approval and a reference for developing the first version of the system
for the development team.

\sectionLabel{Scope}

The ``\projectTitle{}'' is a Windows desktop application which helps blog authors and owners to manage the blog content,
themes, drafts, posts, media and users while offline. The application should be free to download from either the Windows
Store or a direct download link from the software vendor. The software should support at least WordPress, Google Blogger
and Jekyll (GitHub Pages) blogs out of the box. There should be the ability to add external plugins to provide support
for other blog hosting websites and technologies. The plugin ecosystem should be open so as to allow anyone with
supported tools to build an extension for the product. The software should not require administrator permissions to
install or remove to allow it to be used on a per-user installation instead of a system-wide installation.

Furthermore, the software should not need either Internet or GPS connection for any operation expect first time setup,
media upload and post publishing. The software should provide an environment and tools to assist in automating and
bringing the publishing process offline so that all the work can be done offline and pushed online when needed without
the need for constant internet connectivity. The software shouldn't also be limited to managing a single blog at a time
--- a user should be able to add multiple blog accounts and switch between them seamlessly.

\sectionLabel{Definitions, Acronyms, and Abbreviations}

\begin{center}
    \begin{tabularx}{\textwidth}{l L}
        \toprule
        \textbf{Term} & \textbf{Definition} \\
        \midrule
        Software      & The Offline Blog Management Application. \\
        Blog          & An online WordPress, Google Blogger or Jekyll blog hosted on a publically accessible
                        internet address. \\
        Author        & The person submitting an article to be either published or saved as a draft. \\
        Reviewer      & A person that examines an article and has the ability to recommend approval for publication or
                        to request that changes be made to the article. \\
        Stakeholder   & Any person with an interest in the project who is not a developer. \\
        User          & Author or Reviewer. \\
        \bottomrule
    \end{tabularx}
\end{center}

\sectionLabel{References}

\cite{ieee1998ieee}: IEEE Software Engineering Standards Committee, “IEEE Std 830 - 1998, IEEE Recommended Practice for
Software Requirements Specifications”, October 20, 1998.

\sectionLabel{Overview}

This document is intended for casual users, developers, testers and documentation writers, each one having their own
needs and uses of the software. For a more thorough analysis on this matter, please refer to section 2.3. Section 2
provides an overall description of the software, and section 3 describes the functional requirements of the project,
particularly useful to all of the aforementioned groups. Section 4 discusses the external interface requirements, and
finally, in section 5, you can find the non-functional requirements of the project. Each section is divided in
subsections where different matters are discussed.
