\chapterLabel{Non-Functional Requirements}

\sectionLabel{Performance Requirements}

The software should be able to run on any system that allows the installation of the software. That means that any
system with internet connectivity (even if occasional), Windows with .NET Framework 4.6 (Windows 8 and later have it
pre-installed) or Linux with Mono.

As for storage requirements the application should not take more than 50MB in a clean install. The software should be
able to handle arbitrary amount of blog data provided the storage space on the user's system supports it. The software
should also be mindful of the user's storage space and should leave at-least 10\% of the system storage space empty and
alert the user about the situation and proceed only if expressly granted permission by the user.

The software is also not computationally expensive and should hence run on almost any processor as long as the
application can be compiled for the architecture. This generally means that all AMD64, x86-64 and Intel 32-bit
processors are supported.

The software should also be modest about RAM usage and should be able to perform all functions under 200MB of RAM except
for media handling or previewing of a theme.

\sectionLabel{Safety Requirements}

As with any software, data loss is a possibility and hence the software should perform all data bound actions using a
locking mechanism to prevent race conditions and corruption due to simultaneous access of files. The software should
also perform all filesystem based transactions in an atomic manner to ensure that all actions can be rolled back as long
as the backup files are available. The atomic transaction support depends on the filesystem that the software is running
on and hence is not a strict guarantee.

With that said, the software comes with absolutelty no warranty and cannot be held responsible for any loss of data due
to the use of the software.

\sectionLabel{Security Requirements}

The software has mechanisms to ensure that no user is able to perform an action not allowed at their privilege level on
the blog. This means that only those blog accounts that are administrators can actually delete posts, or create posts or
even modify them. The capabilities of non-admin and admin users are not decided by the software but are instead decided
by the blog provider they are using and the configuration settings on the blog providers.

The software is bound by the security features of the host operating system too and hence cannot modify files of another
installation of the same software under a different user account.
