\chapterLabel{Specific Requirements}

This section contains all of the functional and quality requirements of the system. It gives a detailed description of
the system and all its features.

\sectionLabel{External Interface Requirements}

This section provides a detailed description of all inputs into and outputs from the system. It also gives a description
of the hardware, software and communication interfaces and provides basic prototypes of the user interface.

\subsectionLabel{User Interface}

A first-time user of the desktop application should see the first-time setup wizard which is a 7 step wizard to automate
the setting up of a blog for use with the software. The first-time setup wizard will now be referred to as OOBE (for Out
of Box Experience). There should be an option to allow a user to skip the OOBE if they so want but doing so should
provide the user with a detailed message about how to start the OOBE again if needed.

\subsubsectionLabel{Out of Box Experience}

The user should be greeted with an OOBE when starting the software for the first time and also when explicitly invoked
using a menu. The OOBE is a 7 step wizard to help connect an online blog to the software. It consists of the following
steps:

\begin{enumerate}
    \item {Selection of an online blog source. The source can be a WordPress blog (either self-hosted or on
           wordpress.com), a Google Blogger blog or a Jekyll blog (self-hosted or on GitHub pages).}
    \item {Providing the user credentials to the blog to allow connecting to the blog and retrieving information.}
    \item {Customizing operating modes for the blog and giving the source a name to allow easier management of multiple
           sources.}
    \item {Optional step for downloading the theme currently in use at the online blog to allow a one to one local
           preview even without internet connectivity.}
    \item {Optional step for retrieval of comments left on the blog.}
    \item {Creation of a test post on the blog to ensure that the connection works properly. The test post should be
           deleted as soon as the OOBE finishes.}
    \item {Writing the OOBE settings to the Windows registry to ensure that the OOBE runs only once unless explicitly
           invoked.}
\end{enumerate}

\subsubsectionLabel{Normal Operation Interface}

During normal operation the interface should mimic that of Microsoft Office Word. It means that there should be a
ribbon-interface toolbar on the top of the window to provide access to all the menus available in the software. Each
section of the ribbon should be accessible using an intuitive keyboard shortcut. Furthermore, each individual option
within a ribbon panel should also be accessible using nothing but the keyboard.

Below the ribbon interface should be the main text editor. The text editor should have three modes of operation which
should be indicated by three buttons at the bottom of the text editor. The three modes are:

\begin{description}
    \item [Markdown mode:] The text in this mode is the Markdown equivalent of the text entered in the other modes.
    \item [HTML mode:] The text in this mode is the HTML output that will be generated and used to render the post.
    \item [Rich Text mode:] This is the default mode and is similar to how Microsoft Office Word works. The user can
          apply formatting using toolbar buttons and keyboard shortcuts and the changes will be reflected in other
          modes.
    \item [Preview mode:] This is a read-only mode where the user can see the rendered output --- how the post will look
          in a browser when published on the blog. The user needs to have set up the offline theme capability during
          the blog setup process to use this feature. This mode will notify the user to do so if they haven't already
          done it.
\end{description}

\subsubsectionLabel{Management Interface}

A separate menu based interface is used to manage the connected blogs. The interface should be a two-pane interface with
sections on the left side and corresponding information on the right hand side. The main sections should include the
following:

\begin{description}
    \item [Blog Accounts:] This section should list all the currently connected blogs along with the URLs where they are
          hosted and should prominently display identifiable branding so that the user can easily differentiate between
          different blog services. It should provide the following operations:
        \begin{enumerate}
            \item {Viewing details of existing connections.}
            \item {Removing existing connections after confirming the action. Additionally the software should prompt
                   the user to sync information if the local changes have not been synced with the online blog to
                   ensure that no data is lost.}
            \item {Adding new connections. The blog provider branding should be prominently visible to make sure that
                   the user knows which type of blog they are adding.}
            \item {Changing blog connection credentials. It is possible that the blog's user authentication credentials
                   have been changed after setting up the blog (due to the user changing their password). To help fix
                   this issue the user can simply change the stored password and username from this menu to avoid the
                   need to reconnect to the blog from scratch.}
            \item {Explicitly starting the OOBE. This menu should provide a button to explicitly start the OOBE if the
                   user skipped it during the first-time startup of the application.}
        \end{enumerate}
    \item [Plugins:] This section should provide a list of currently installed plugins with a button to open the
          plugin's settings. There should also be a button to list all available plugins. The list will be retrieved
          from the internet and cached until the next time internet connectivity is detected.
    \item [Text Editor:] This section will list all the settings pertaining to the text editor including the following:
        \begin{enumerate}
            \item {Default mode for the text editor.}
            \item {Default font used for the Markdown mode.}
            \item {Markdown flavour used. The software should handle CommonMark, GitHub Flavoured Markdown and both
                   Setext and ATX style headers.}
            \item {Enabling and disabling of syntax highlighting if code is detected.}
            \item {Word wrap and other similar features.}
        \end{enumerate}
    \item [Internet Settings:] This section allows the configuration of the following:
        \begin{enumerate}
            \item {Whether to use a proxy server to connect to the internet.}
            \item {Proxy server to use for HTTP connections.}
            \item {Proxy server to use for HTTPS connections.}
            \item {Maximum bandwidth the product is allowed to use.}
        \end{enumerate}
    \item [Update Settings:] This section allows the user to check for any available updates or to switch from the
          regular update channel to nightly releases channel.
\end{description}

\subsectionLabel{Hardware Interface}

Since neither the desktop application nor the extensibility API have any designated hardware, it does not have any
direct hardware interfaces.

\subsectionLabel{Software Interface}

The desktop application communicates with the web server where the blog is hosted in order to get the data from the blog
including previous posts, drafts, comments and associated metadata.

\subsectionLabel{Communication Interface}

The communication between the different parts of the system is important since they depend on each other. However, in
what way the communication is achieved is not important for the system and is therefore handled by the underlying
operating systems for both the desktop application and the extensibility API.

\sectionLabel{Functional Requirements}

This section includes the requirements that specify all the fundamental actions of the software system.

\subsectionLabel{Download}

The user should be able to download and install the desktop application from an application store or a direct download
link.

\subsectionLabel{Update delivery and notification}

The software should notify the user of any available software updates and download and apply them according to the
settings user has configured in the application.

\subsectionLabel{Connecting a blog with the software}

The software should allow a user to connect a publicly accessible blog to the software.

\begin{description}
    \item [Inputs:] ~
    \begin{enumerate}[noitemsep]
        \item {The web address of the blog.}
        \item {The type of blog or the service that it is hosted on. Can be one of Google Blogger, WordPress, Jekyll or
               GitHub Pages.}
        \item {User credentials for authenticating to the website. Google Blogger additionally requires that API access
               is enabled in Google Blogger settings.}
    \end{enumerate}
    \item [Process:] ~
    \begin{enumerate}[noitemsep]
        \item {Send an HTTP GET request to the web address to ensure that the address exists.}
        \item {Check that the type of blog at the web address matches the type specified by the user.}
        \item {Download the files from the blog including theme, content and metadata.}
        \item {Create a test post on the blog using the application and check if it was created correctly to verify that
               the connection is working.}
    \end{enumerate}
    \item [Error Handling:] ~
        \begin{enumerate}[noitemsep]
        \item {In the case of absence of a website at the URL we present a message to the user to fix the URL.}
        \item {In the case of mismatch between user specified blog type and detected blog type we present a message a
               message to the user with the option of automatically fixing their mistake.}
    \end{enumerate}
\end{description}

\subsectionLabel{Creating a new post}

The software should allow the user to create a new blog post.

\begin{description}
    \item [Inputs:] ~
    \begin{enumerate}[noitemsep]
        \item {The title of the blog post.}
        \item {The category of the blog post.}
        \item {The published state of the post. It can be published or draft.}
        \item {The blog connection on which the post is to be made.}
        \item {The associated metadata specific to every blog provider.}
    \end{enumerate}
    \item [Process:] ~
    \begin{enumerate}[noitemsep]
        \item {Create a file on the local filesystem in the application's data directory representing the post heirarchy
               on the associated blog account.}
        \item {Create a metadata file of the same name as the post with a different extension to keep track of
               associated post metadata.}
        \item {Present the post file in the application's editor. The default mode should be rich text editing mode.}
    \end{enumerate}
    \item [Error Handling:] ~
        \begin{enumerate}[noitemsep]
        \item {If a post by the same name already exists on the same blog, provide an alert to the user and allow them
               to change the name without having to create another post from scratch.}
        \item {If a post by the same name already exists on the same blog but is not synced locally, alert the user the
               next time the application tries to sync data from the online blog.}
    \end{enumerate}
\end{description}

\subsectionLabel{Rename an existing post}

The software should allow the user to rename an existing post.

\begin{description}
    \item [Inputs:] ~
    \begin{enumerate}[noitemsep]
        \item {The title of the blog post to rename.}
        \item {The category of the blog post.}
        \item {The published state of the post. It can be published or draft.}
        \item {The blog connection to which the post belongs.}
        \item {The new name to give to the blog post.}
    \end{enumerate}
    \item [Process:] ~
    \begin{enumerate}[noitemsep]
        \item {Check if a file corresponding to the new name already exists in the application's data directory. If it
               does, alert the user and allow them to change the name without starting over.}
        \item {Create a new file on the local filesystem and copy the contents of the current file over to the new
               file. This should include the metadata and url files as well.}
        \item {Delete the files with the old name only after ensuring that the newly created files can be read correctly
               and there is no difference in the content.}
    \end{enumerate}
    \item [Error Handling:] ~
    \begin{enumerate}[noitemsep]
        \item {A file with the new name already exists then alert the user and allow them to change the name without
               having to start the rename process again.}
        \item {If a post by the same name already exists on the same blog but is not synced locally, alert the user the
               next time the application tries to sync data from the online blog.}
        \item {The rename failed due to some other reason then transparently roll-back the changes and inform the user
               about it. Allow them to either retry the action, abort the action or look at the logs to allow submitting
               them to the developers for further investigation.}
    \end{enumerate}
\end{description}

\subsectionLabel{Modify an existing post}

The software should allow the user to modify an existing post including it's title, content, metadata, media, published
status and publish date.

\begin{description}
    \item [Inputs:] ~
    \begin{enumerate}[noitemsep]
        \item {The title of the blog post to modify.}
        \item {The category of the blog post.}
        \item {The published state of the post. It can be published or draft.}
        \item {The blog connection to which the post belongs.}
    \end{enumerate}
    \item [Process:] ~
    \begin{enumerate}[noitemsep]
        \item {Open a new window with two panes, the left one for editing the visual content (title, body, media etc.)
               and the right pane to edit the metadata like post category, published status, date, author etc.}
        \item {Allow the user to edit any and all of those fields freely.}
        \item {Create an autosave of the new information as backup every 1 minute.}
        \item {Prompt the user to save the information when closing the window if any changes were made.}
        \item {Saving the changed information should edit the backing files in-place instead of creating new copies to
               avoid overhead.}
    \end{enumerate}
    \item [Error Handling:] ~
    \begin{enumerate}[noitemsep]
        \item {A file with the new name already exists then alert the user and allow them to change the name without
               having to start the modify process again.}
        \item {If a post by the same name already exists on the same blog but is not synced locally, alert the user the
               next time the application tries to sync data from the online blog.}
        \item {The modification failed due to some other reason then transparently roll-back the changes and inform the
               user about it. Allow them to either retry the action, abort the action or look at the logs to allow
               submitting them to the developers for further investigation.}
    \end{enumerate}
\end{description}

\subsectionLabel{Add media to a post}

The software should allow the user to attach media files of any type as long as they are supported by the blog provider.

\begin{description}
    \item [Inputs:] ~
    \begin{enumerate}[noitemsep]
        \item {The title of the blog post to add the media to.}
        \item {Optional: Location within the document where to attach the media file.}
        \item {The list of filetypes accepted by the blog provider.}
        \item {The URL or local fileystem URI to the media to attach.}
    \end{enumerate}
    \item [Process:] ~
    \begin{enumerate}[noitemsep]
        \item {Open a two paned window with a URL bar and browse button on the top to allow for manually providing a URI
               to fetch the media files from.}
        \item {The left pane shows previews of the media file currently selected.}
        \item {The right pane allows modifying the properties of the media file before inserting it into the blog post.}
        \item {If internet connectivity is available then the software should fetch the media files from remote URLs
               otherwise it should simply write the URL to the URL file associated with the blog post so that the media
               can be retrieved at a later time.}
    \end{enumerate}
    \item [Error Handling:] ~
    \begin{enumerate}[noitemsep]
        \item {If the filetype is not supported by the blog provider alert the user about the same and show them a list
               of accepted filetypes.}
        \item {If the remote media could not be fetched inform the user and ask them if to retry or write the
               information to the associated URL file for later updatation.}
    \end{enumerate}
\end{description}

\subsectionLabel{Delete a post}

The software should allow the user to delete any existing posts as long as the user has sufficient priviledges.

\begin{description}
    \item [Inputs:] ~
    \begin{enumerate}[noitemsep]
        \item {The title of the blog post to delete.}
        \item {The user priviledge level on the blog.}
    \end{enumerate}
    \item [Process:] ~
    \begin{enumerate}[noitemsep]
        \item {Show a confirmation message asking the user to confirm the action. If the user accepts it then delete the
               post but keep a local copy of it for at least 30 days.}
        \item {Move the data files, metadata and url files to a specific directory used to store deleted posts.}
    \end{enumerate}
    \item [Error Handling:] ~
    \begin{enumerate}[noitemsep]
        \item {The user doesn't have sufficient privileges for the action then alert the user asking them to contact an
               administrator or log in using the blog administrator's account.}
    \end{enumerate}
\end{description}

\subsectionLabel{Schedule a blog post}

The software should allow a user to schedule a blog post to be published at a future time and date.

\begin{description}
    \item [Inputs:] ~
    \begin{enumerate}[noitemsep]
        \item {The blog entry to schedule.}
        \item {The date and time on which to schedule the post.}
    \end{enumerate}
    \item [Process:] ~
    \begin{enumerate}[noitemsep]
        \item {Show a dialog box telling the date and time both in local timezone and UTC timezone to allow the user to
               confirm if the times are correct. If the user accepts schedule the post otherwise abort the action.}
        \item {If the blog provider supports scheduled posts (WordPress and Blogger) use their API to schedule a post.}
        \item {If the blog provider doesn't support scheduled posts then use the application's own scheduler to create a
               scheduled job entry and notify the user that the scheduled post will only be posted if the user has the
               system turned on with internet connectivity.}
    \end{enumerate}
    \item [Error Handling:] ~
    \begin{enumerate}[noitemsep]
        \item {If the scheduled time is missed due to the system being turned off or no internet connectivity, wait for
               internet connectivity and update the post after informing the user of the situation.}
    \end{enumerate}
\end{description}
