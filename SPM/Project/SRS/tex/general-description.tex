\chapterLabel{General Description}

This section will give an overview of the whole system. The system will be explained in its context to show how the
system interacts with other systems and introduce the basic functionality of it. It will also describe what type of
stakeholders that will use the system and what functionality is available for each type. At last, the constraints and
assumptions for the system will be presented.

\sectionLabel{Product Perspective}

This system will consist of two parts: one Windows desktop application and one extensibility API for writing plugins
that interact with the desktop application. The desktop application will be used to manage blogs while the extensibility
API will be used for writing plugins that modify behaviour of the desktop application or add additional features. The
desktop application will need to communicate to a web server which hosts the blog over the internet during it's first
time setup and additionally when publishing content or retreiving media files. The product will be written using C\# and
the .NET framework and hence expects that the machine where the software will be run is a Windows machine with .NET
Framework 4.6 available or a Linux machine with Mono or dotNET Core installed.

There are similar products which allow blog management but most are limited to a single blog software like PHP WordPress
Dashboard, Ghost Desktop App, Hootsuite etc. and all of them invariably need internet connectivity for proper
functioning.

\sectionLabel{Product Features}

A summary of product features follows:

\begin{itemize}
    \item {Connect to online blog accounts which use WordPress, Google Blogger or Jekyll.}
    \item {Create new blog posts including post text, images, categories and additional metadata.}
    \item {Create new draft posts server side which are published by the server. Only WordPress supports this.}
    \item {Create new local draft posts which can be scheduled and synced automatically provided Internet is available
           and the machine is turned on at the scheduled time.}
    \item {Locally preview posts and drafts using the same theme as on the remote blog.}
    \item {Manage blog comments including writing, deleting or marking a comment for review. Only Google Blogger and
           WordPress support this.}
    \item {Periodic sync to ensure that the data on the online blog and the local data are not inconsistent with
           automatic resolution of inconsistensies using heurestics.}
    \item {Addition of a secondary blog in addition to the primary blog required to use the software.}
    \item {Export of all blog posts and drafts from an existing blog to an XML based format or plaintext format
           depending on further needs.}
\end{itemize}

\sectionLabel{User Characteristics}

There are two types of users that interact with the system: users of the desktop application, and users of the
extensibility APIs. Each of these types of users have different uses of the system so each of them have their own
requirements. The desktop application users can only use the application to manage the blogs using functions discussed
in section 2.3 above. The extension authors will not use the desktop application directly but can instead interact with
it programmatically using the provided extensibility API. The extensibility API should ideally be wrapped using a
library that provides binding in popular languages like C\#, C++, Ruby and Python but that will be subject to feature
prioritisation during the development of the product.

\sectionLabel{General Constraints}

The desktop application is constrained by the system resources only. The biggest constraint is the availability of
internet for the first time setup of the software and during media retreival. Another third party constraint is the
availability of the web server where the user's blog is hosted. The software is also somewhat limited by the locale that
the user uses their system in because there are no localizations planned for the initial version of the software meaning
that the entire interface will use English US as the interface language.

\sectionLabel{Assumptions and Dependencies}

One assumption about the product is that it will always be used on desktop machines that have a means of running the
.NET Framework either natively or using a host system like Mono. If the desktop does not have the .NET Framework
available then the software will simply refuse to install. Another assumption is the absence of any firewall
restrictions for the application or the presence of a sane proxy to use the internet for all operations that require
internet connectivity.

\sectionLabel{Apportioning of Requirements}

In the case that the project is delayed, there are some requirements that could be transferred to the next version of
the application. Those requirements are to be developed in the second release. See the revised versions of the SRS for
reference.
